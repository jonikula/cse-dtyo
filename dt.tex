%
% Template for Department of Electrical and Information Engineering Diploma Thesis v1.1.2013
% Authors: Mika Korhonen (original author), Pekka Pietikäinen, Christian Wieser, Teemu Tokola and Juha Kylmänen.
% If you make any improvements to this template, please contact ouspg@ee.oulu.fi
%

\documentclass[a4paper, 12pt,titlepage]{dithesis}
\usepackage[english,finnish]{babel}
\usepackage[utf8]{inputenc}
\usepackage[T1]{fontenc}
\usepackage{times}
\usepackage{tabularx}
\usepackage{graphicx}
\usepackage{float}
\usepackage{enumerate}
\usepackage{placeins}
\usepackage{fancybox}
\usepackage{verbatim}
\usepackage{longtable}
\usepackage{di}
\usepackage[hyphens]{url}
\usepackage{boxedminipage}
\usepackage{subfigure}
\usepackage{multirow}
\usepackage{amsfonts}
\usepackage{xcolor}
\tolerance=500

%\usepackage[a4paper,margin=2.5cm,dvips]{geometry}
%\geometry{papersize={210mm,297mm}}
%dvipdf -sPAPERSIZE=a4

% The following code removes %-signs with URL:s longer than 72 chars
\begingroup
\makeatletter
\g@addto@macro{\UrlSpecials}{%
  \endlinechar=13 \catcode\endlinechar=12
  \do\%{\Url@percent}\do\^^M{\break}}
 \catcode13=12 %
 \gdef\Url@percent{\@ifnextchar^^M{\@gobble}{\mathbin{\mathchar`\%}}}%
\endgroup %

%\selectlanguage{finnish}

\otsikko{Mallipohjainen testaustyökalu Asioiden Internet-järjestelmälle}
\title{Model-Based testing tool for Internet of Things systems}

\etunimi{Joonas}
\sukunimi{Nikula}
\valvoja{prof. Susanna Pirttikangas}
\koulutusohjelma{information} % {information | electrical}
\vuosi{2017}
\tyo{Master} % {Bachelor | Master}
\kieli{english} % {finnish | english}

\begin{document}

%% Computer science and engineering
%\iffalse     % comment multiple lines with \iffalse, \fi -pair
    \begin{titlepage}
        \centering{\includegraphics*[width=0.3\textwidth]{uni_logo}\\}
        {\sffamily\fontsize{9}{1pt}\selectfont FACULTY OF INFORMATION TECHNOLOGY AND ELECTRICAL ENGINEERING\\}
        %{\sffamily\fontsize{9}{1pt}\selectfont TIETO- JA SÄHKÖTEKNIIKAN TIEDEKUNTA\\}
        \vspace{65 mm}
        {\textbf{\fontsize{16}{19pt}\selectfont \getfirstname\ \getlastname }\\}
        \vspace{15 mm}
        {\textbf{\fontsize{18}{22pt}\selectfont MODEL BASED TESTING TOOL FOR IOT APPLICATIONS\\}}
        \vspace{60 mm}
        {\fontsize{14}{17}\selectfont Master's Thesis \\Degree Programme in Computer Science and Engineering \\ October 2017\\}
        %{\fontsize{14}{17}\selectfont Diplomityö \\Tietotekniikan tutkinto-ohjelma \\ Month 20xx\\}
    \end{titlepage}
%\fi

%% Electrical engineering
\iffalse     % comment multiple lines with \iffalse, \fi -pair
    \begin{titlepage}
        \centering{\includegraphics*[width=0.25\textwidth]{uni_logo}\\}
        {\sffamily\fontsize{8}{10}\selectfont DEGREE PROGRAMME IN ELECTRICAL ENGINEERING\\}
        %{\sffamily\fontsize{8}{10}\selectfont SÄHKÖTEKNIIKAN KOULUTUSOHJELMA\\}
        \vspace{50 mm}
        {\textbf{\fontsize{20}{24}\selectfont MASTER'S THESIS }\\}
        %{\textbf{\fontsize{20}{24}\selectfont DIPLOMITYÖ }\\}
        \vspace{18 pt}
        {\textbf{\fontsize{18}{22}\selectfont TITLE IS WRITTEN HERE \\ ADDITIONAL LINE IF NEEDED \\}}
        \vspace{60 mm}
        {\fontsize{12}{15}\selectfont
            \begin{table}[ht]
                \centering
                \begin{tabular}{p{4.5cm}l}
                    Author & \getfirstname\ \getlastname \\ \\
                    %Tekijä & Etunimi sukunimi \\ \\
                    Supervisor & Firstname Surname \\ \\
                    %Valvoja & Etunimi sukunimi \\ \\
                    Second Examiner & Firstname Surname \\ \\
                    %Toinen tarkastaja & Etunimi sukunimi \\ \\
                    (Technical advisor & Firstname Surname) \\
                    %(Työn tekninen ohjaaja & Etunimi sukunimi) \\
            \end{tabular}
        \end{table}}
        {\fontsize{12}{15}\selectfont Month 20xx}
    \end{titlepage}
\fi


\selectlanguage{english}

\begin{abstract}
In this thesis I will study model based testing (MBT) from the viewpoint of a testing tool developer
and present a prototype of a testing tool or a solution based on an existing testing tool that fulfills 
a set of requirements described later in this thesis. The tool is meant to be a prototype for generating tests 
for a network of IoT devices with a high level of abstraction.    

\keywords Testing, Model Based Testing, Software Development, Internet of Things

\end{abstract}

\selectlanguage{finnish}
\begin{tiivistelma}
Esimerkkitiivistelmä

\avainsanat Testaus, Mallipohjainen testaus, ohjelmistokehitys, esineiden internet
\end{tiivistelma}

\selectlanguage{english}
%\selectlanguage{finnish}

\sisluettelo
%\tableofcontents

\otsake{FOREWORD}
Lorem ipsum dolor sit amet, consectetur adipiscing elit. In sagittis eleifend dui sit amet luctus. Nullam vitae turpis id sapien viverra tincidunt quis id turpis. Etiam eget sodales eros. Cras suscipit porttitor odio a scelerisque. Proin et velit sed dolor gravida tincidunt. Nullam quis suscipit mi. Pellentesque id consectetur mauris, ac bibendum mi. Nunc tincidunt ac augue vitae elementum.

Sed et semper quam. Praesent et tincidunt nunc. Aliquam facilisis egestas urna molestie hendrerit. Duis eu nisl ultrices, pretium dui ut, euismod lorem. Cras mattis justo id laoreet vehicula. Donec dictum finibus lorem. Quisque posuere, dui sit amet feugiat luctus, sapien nunc interdum nibh, nec pulvinar lectus nisl id tortor. Cras eget hendrerit metus.
%\allekirjoitus{Oulu, Finland \today}

\otsake{ABBREVIATIONS}

\setlongtables
\begin{longtable}[l]{p{3cm}p{0.7\textwidth}}

% Add your abbreviations to abbreviations.tex
IoT & Internet of Things\\
MBT & Model-based testing\\
SUT & System under test / Software under test\\

\end{longtable}
\setcounter{table}{0}

%Johdanto
\chapter{Introduction}
\sivunumerot
\thispagestyle{empty}
According to the Guide to the Software Engineering Body of Knowledge, the view of software testing has matured into a constructive one during the last few years. It is no longer seen as an activity that starts only after the coding phase is complete. Instead, it has become an activity that is pervasive throughout the development and maintenance cycle of the software \cite{swebok}. 

As the importance of testing grows, the methods and techniques used in testing need to evolve and grow as well. As the complexity and nature of the software under test grows, many previous methods of testing are becoming difficult to scale up at the same pace. Manual testing is too resource-intensive and different forms of capture and replay testing are too expensive to maintain with the modern, agile development processes. Many companies have transitioned into using automated testing with scripts executed periodically or on triggers by test harnesses. This is a logical step forward, but even this can sometimes be insufficient on its own, as the resource requirements for developing and maintaining these test scripts grows rapidly with the size of the software and the size of the test suite. Model-based testing pushes the level of automation further by automating the design, not just the execution, of the test cases \cite{tools}. 

Model-based testing is a test technique that has evolved from automated testing techniques via scripts and test harnesses. As the name suggests, model-based testing is built around the concept of a model, an abstract, formal representation of the SUT or of its software requirements. MBT is used to validate requirements, check the consistency of requirements and generate test cases focused on the behavioural aspects of the software. It is most often used in conjunction with test automation harnesses \cite{swebok}.

 % ./introduction.tex

\chapter{Model-based Testing}
This chapter outlines the key issues of software testing as well as previous testing methods and their shortcomings.
After identifying these key issues and problems, the solutions offered by adopting model-based testing are presented.
The model-based testing process is then studied further. The steps from model generation to analysis of results are presented 
before focusing more on the key part of the testing process: the model. The text in this chapter is based largely on The Guide to the Software Engineering Body of Knowledge \cite{swebok} and Practical Model-Based Testing: A Tools Approach. \cite{tools}

\section{Software testing}
The Guide to the Software Engineering Body of Knowledge defines software testing as the dynamic verification that a program provides expected behaviors on a finite set of test cases, suitably selected from the usually infinite execution domain. From this definition, the authors extracted four words that describe the software testing knowledge area: dynamic, finite, selected and expected.

The authors use of the word dynamic to describe testing excludes the static analysis of the program from the field of testing, although it is described as complementary and different to dynamic testing. To be precise, the term dynamic is used to highlight the fact that even though testing always implies executing the program with selected inputs, the input values alone are not always sufficient to specify a test. This is due to the fact that complex, nondeterministic systems might react differently to the same input, depending on the state the software and the environment.
Actually executing the program during testing in a real or simulated environment is a significant advantage. The compiler, the libraries, the operating system and many other parts of the solution are tested alongside the correctness of the software design and code.

Testing is finite by necessity. Even in simple programs, so many possible test cases exist that covering them all could require more time and resources than is practical. For this reason, testing is conducted on a subset of tests, determined by risk and prioritization criteria. This also highlights the issue of test selection.

Test selection is a complex problem. It is also one of the key challenges in testing. There are many proposed test techinques and software engineers need to be aware of their effectiveness in a given situation. There are several informal strategies, such as equivalence class and boundary value testing that can help in determining the tests that are the most likely of being effective, and some of these techniques can also be applied in model-based testing.

The final key word, expected, is related to the oracle problem. After test execution, the result need to be analysed and a decision on the correctness of the behaviour needs to be made. The observed behaviour can be checked against user needs, specifications, anticipated behaviour and other kinds of requirements and expectations. Model-based testing can also be applied to generation of oracles as well as test inputs.

\section{Definition of model based testing}
The different kinds of testing can be described using a three-axis model presented by Tretmans \cite{tretmans}.  This model can be used to describe model-based testing as well. In the model, the axis represent the scale of the tested component, the source the tests are derived from and the characteristics being tested. The scale of the tested component runs all the way from software units, such as functions and classes, to individual components of software, integration testing and system testing. The second axis, characteristics, contains functional testing, robustness testing, performance testing and usability testing. The third axis is split into two by black-box and white-box testing. Using this model, model-based testing can be described as functional black-box testing and it can be applied to the entire SUT scale. There are some applications of model-based testing in performance and robustness testing as well, but especially performance testing methods for model-based testing are still under development.


The following approaches are the ones usually known as model-based testing.
 
\begin{enumerate}
	\item Generation of test input data from a domain model
	\item Generation of test cases from an environment model
	\item Generation of test cases with oracles from a behavior model
	\item Generation of test scripts from abstract tests
\end{enumerate}

Each of these can be said to be model-based testing, but some of them are not quite suitable when model-based testing is viewed in the context of tool development. In this context, the third definition is the one that draws the focus.
The first definition is an integral part of model-based testing, but it does not solve the entire problem of test design because it does not provide information of whether a test has passed or failed. Similarly to the first, the second definition does not provide information on the success or failure of the test case in any useful capacity, since it does not model the behaviour of the SUT. Rather it focuses on the environment around the SUT. The fourth definition is very different from the other three definitions. It assumes that an abstract description of a test case exists, perhaps in the form of an UML sequence diagram. This description is then transformed into an executable test script.

The third definition, the generation of test cases with oracles from a behavior model, is the one that gives us the most useful description of model-based testing. With this definition, the complete testing process can be automated, given a suitable model, and it produces test sequences that are suitable for transformation into executable test scripts.

With this view of model-based testing, Utting and Legeard define model-based testing as automation of the design of black-box tests.

\section{Testing methods and key issues}
As established earlier, model-based testing is usually applied to functional testing. For functional testing, Legeard and Utting present the following three key issues:

\begin{enumerate}
	\item Design of test cases
	\item Test execution and analysis of results
	\item Verifying how the tests cover the requirements
\end{enumerate}

There are several classic testing processes that are widely used and that try to address these issues. These processes are manual testing, capture/replay testing, script based testing and keyword-driven automated testing.

\subsection{A Manual Testing Process}
An entirely manual testing process is still widely used in the industry, even though it is the earliest and simplest for of testing. In manual testing, the test designer creates a set of test cases based on the requirements documentation and the objectives, strategies and other information presented in a test plan. Test design is entirely manual and it's output is a human-readable document that describes each test case. The level of precision in these kinds of test case descriptions can be very low, as the low-level details regarding the SUT interaction can be left to the common sense of the tester. This type of test case design is very time consuming and it generally does not ensure systematic coverage of the functionality.

Here test execution is also manual. The manual tester executes each test case by going through the steps of that test case, interacting directly with the SUT or with a test execution environment and analysing the output before recording the verdict for that test case.

This process is repeated every time a new release needs to be tested, and as is apparent from the process, it can be quite time consuming and resource intensive. The cost of testing each release is directly related to the size of the test suite and the cost of testing is constant as long as the size of the suite remains the same. This often results in the need to reduce the number of test cases being run as the software evolves, which introduces significant risks regarding product maturity, stability and robustness.

\subsection{Capture/replay Testing Process}
Capture/replay testing is a way to add automation to the manual testing process. The initial steps for this process are similar to that of the manual testing process. The test engineer creates a set of test cases, which he then executes manually. While executing the tests, the engineer uses a tool to capture the steps he performs on the software. These captured steps can then be automatically executed, removing the need to manually re-execute tests that have been previously captured.

The idea here is solid, but it suffers similar problems as the manual testing process. It still initially requires lots of manual work to execute the test cases. The captured tests need to also be re-executed manually every time the tested system changes as the captured test might no longer work on the new version of the software. In this case the tests need to be manually redone and captured again to get the automation to work with the new software.

\subsection{Script-based Testing Process}
A better form of automating testing is based on test scripts. These scripts are used to execute the test on the SUT and they can be automatically triggered for execution on different kinds of triggers, such as time or a commit done on a version control system. In this testing process the test engineer creates test scripts for each of the test cases using a programming or scripting language of some sort and adds them to the test suite. These tests are then automatically executed when necessary. This testing process requires very little manual testing after the test scripts have been created. Some manual testing may be done while developing the test scripts to verify functionality and to help in the development of the scripts.

After the test scripts have been created, only minimal maintenance work needs to be done to make sure the scripts are up to date with the current version of the tested software. Manual work is still needed when initially creating the test scripts but after the initial investment in time the maintenance cost is significantly reduced.


-- Solved and remaining problems

\section{Model based testing process}
TÄHÄN LISÄTÄÄN KUVA

-- Main idea is to take the informal, mental model of expected SUT behaviour and create a formal model that enables automatic tc creation. 

-- Five main steps
	-- Model SUT and/or it's environment
	-- Generate abstract tests
	-- Concretize abs. tests into executable tests
	-- Execute tests and assign verdicts
	-- Analyze results

\subsection{The model}
-- Must be concice and precise. 

-- Most essential thing in MBT.

-- Problem: Using development models or develop entirely own models for testing? 
	-- usually a good middle ground, reuse of high-level diagrams and some use cases from dev. models and adding behavioural details nexessary for MBT.

-- Work must be spent on developing initial model, but pays off in lower maintenance costs.
	-- requirements evolution handled by evolving model, new tc:s then generated automatically from model! 
	
\subsubsection{Model design process}
-- First decide on level of abstraction

-- CD of SUT parts relevant to model is a good starting point.

-- Useful to create smaller models for subsystems instead of large, complex models for the entire system. 

-- Decide on which parts to model

-- identify data and operations required for modeling, with primary focus on SUT.
	-- Omit everything that is unnecessary for the model and include only the operations you wish to test.
	
-- Design model to meet test objectives!

-- Model operations do not need to copy SUT operations, can instead correspond to a sequence of operations or a subset of a single more complex SUT operation!

-- Select notation

\subsection{Advantages and disadvantages}

\subsubsection{Advantages}
-- Saves resources after initial training period when compared to other methods

-- Similar or better performance in finding issues in implementation

-- Lots of issues found in requirements due to precision required in model generation

-- Improved coverage

-- Promotes maturity in SUT development process

-- Traceability from test cases to requirements

\subsubsection{Disadvantages}
-- Requires reasonable maturity from SUT development process and testing processes, as model generation requires proper requirements for SUT and well made models in requirements help reduce time required to generate models for testing.

-- Also requires experience with automated test execution

-- Requires different skills from testers: Modeling skills from test designers, programming skills. Implies some training costs and an initial learning curve.

-- Pain factors after adopting MBT: Outdated requirements cause lots of false errors. Eperience required from testers to recognice where modeling SUT may be difficult and where manually designed test cases might be a better approach. Generated test cases might not always be intuitive to analyze compared to manually created ones. Metrics like "number of test cases run" become useless as it is trivial to generate enormous test suites.

\section{Test selection criteria}
-- Means of communicating choice of tests to tool

-- Not necessary to have knowledge of SUT code to generate tests: Instead based on model and requirements.

-- Model vs. SUT coverage criteria are complementary, use both!

-- Coverage criterion can be used to measure suiote adequacy or to define a stopping point for test generation.

-- Also prescriptive from tool point of view: "Try to achieve X coverage."

-- Families: Structural model coverage, data coverage, fault-model, requirements-based, explicit, statistical

\subsection{Structural model coverage}
-- Major issue in MBT is to measure and maximize coverage of model

-- Control-flow oriented, data-flow oriented, transition-based, UML-based

\subsubsection{Control-flow oriented coverage criterion}
--  Derived from classical code coverage criteria.

-- Statement-, decision-, and path coverage, where statement coverage is the weakest and path coverage the strongest.

-- Condidion coverage, decision/condition coverage, full predicate coverage, modified cond./dec. coverage, multiple condition coverage.

\subsubsection{Transition-based coverage criterion}
-- Useful for SUT:s and models that can be represented as FSMs or EFSMs (or LTS).

-- Transition based models made up of transitions and states.

-- All-transitions coverage, all-states coverage, all-configurations coverage, all-transition-pair coverage, all loop-free paths, all one-loop paths, all round-trips, all-paths.

-- Configuration: Current state of a parallel statechart.

-- All round-trips more usually more useful than all one-loop paths. It is weaker, but gives a linear number of tests (all one-loop paths usually exponential). TÄHÄN LÄHDE Testing object oriented systems: ... R.V. Binder.

-- Usually a good minimum requirement for FSM type models is to produce suite with all-transitions coverage requirement.

\subsection{Data-flow oriented coverage}
-- Working with definitions and uses of data variables

-- All-definitions, all definition/use paths, all use paths.

\chapter{Implementing something}
% Implementation Chapter
The implementation of the prototype was done in 4 main phases, which are discussed in this chapter. First, some high-level requirements and objectives were identified with the customers representatives. Based on these high-level requirements as well as the example set by an existing open-source model-based testing tool, a proof of concept (PoC) implementation was created.


-- Desing refined and specified after first PoC.

-- Implementation.

-- Briefly on testing plan.


\section{Requirements specification}

\subsection{Product scope}
The software system will be a prototype of a model-based testing tool for the testers in ARM system testing team. It will be designed so that starting its use will be easy for the testers, requiring as little training as possible. It will include state-machine style modeling support, including relevant test selection criteria. 
The tool is designed so that it will be easily utilized as a part of the system testing teams existing tools. It will contain abstract interfaces for adapters used to generate the executable tests from abstract test cases generated by this tool. The software will be restricted to the abstract level of test case generation.

\subsection{Product perspective}
The tool developed for this thesis is a model-based testing tool designed to fit into the model-based testing process flow. It covers the first steps in the process: the model and the test case generator as well as an interface to enable use of test script generators which plug in to existing testing tools used by the organization.

\subsection{Product functionality, users and operating environment}
This section summarizes the major functions the software must perform or must let the the different actors perform through various use cases.

\subsubsection{Users and characteristics}
The normal user of the software is expected to be a test engineer in the ARM system testing team. They are expected to have basic skills in Python, experience in testing and formal modeling as well as a basic understanding of programming. 

The test engineer will be able to perform at least the following high-level use cases with the software. 

\begin{enumerate}
	\item Develop a model
	\item Maintain a model and related test suites
	\item Analyze requirements traceability matrix and coverage reports
	\item Add specific, custom test into the suite
\end{enumerate}  

-- Add brief descriptions of each use case

\subsubsection{Operating environment}
The main component of the tool will be a software application, which is usable in any modern operating system that has Python 2.7. The software is designed with a command line interface and so that it requires no external tools to be used. The tool will also need to be able to interact with current testing tools and adapters. 

\subsection{Design and implementation constraints}
There are very few constraints for this piece of software. Primarily it will need access to the file system to be able to generate test cases and read model files. The system also requires Python 2.7 or newer to run. The test case generation itself should not require access to internet or internal networks, but executing the test cases might require these. The tool should also be able to upload select parts of the files and data it generates into an online database. In addition to these minor constraints, the only major design constraint is the user interface. The command line interface needs to be taken into account in the design, as it places constraints on the usability of the system.

\subsection{User documentation}
All the major on minor functionalities will be documented in Markdown format in a GitHub repository, alogn with some easy-to-use tutorials and examples. UML documents describing the different parts of the system will be made available to the users as well as a part of this thesis.  

\subsection{Interfaces}
The interfaces required and presented by the tool are defined in this section. 

\subsubsection{User interfaces}
The tool is accessible to users through a command line interface, which uses simple, sharthand options to guide the functionality of the tool. The command flow should be split into two successive main branches, which can also be run in succession automatically. Several minor functionalities should be accessible along these two main flows.

\subsubsection{Hardware interfaces}
The tool is a pure software tool so it has no hardware interfaces. 

\subsubsection{Software interfaces}
The application is to be developed with Python 2.7 in mind, but support for Python 3 should be maintained where possible. The tool will use interfaces provided by Python to interface with the operating system and the file system when generating test cases, log files and reports. The tool should also have access to an online database via HTTP (described further in communications interfaces).

\subsubsection{Communications interfaces}
The application will use HTTP/HTTPS to communicate with an online database service.

\subsection{Other non-functional requirements}

\subsubsection{Performance requirements}
The performance of the system is largely related to user decision, as the size of the suite generated is directly related to the speed of the execution. In general, the application should not take more than a minute or two to generate a moderately sized test suite. Just in case, the tool should report it's progress in the command line so that the user can follow the execution. The tool should not cause any major performance issues on the computer running it either. There should be no impact on the performance of the actual test runs, since the tool is designed for offline test generation, where the tests are not executed during generation.

\subsubsection{Safety and security requirements}
This application will be able to post data into an online database, which includes some security risks. These should be handled with proper sanitation of the data on the server side as well as use of secure communications channels wherever possible.

\subsubsection{Software quality attributes}
The command line interface of the application should be designed to be simple and powerful and it should provide sufficient feedback and status reports to the users. Possible errors should be easily identified. The test cases generated by the tool should be human readable and should reflect the selection criteria, fulfilled requirements and the actual test case that the script implements.

\section{Design}

\section{Implementation}

\section{Testing}  % ./implementation.tex

\chapter{Testing something}
\input{testing}  % ./testing.tex

\chapter{Discussion}
\input{discussion}  % ./discussion.tex

\chapter{Conclusion}
\input{conclusion}  % ./conclusion.tex

\bibliographystyle{di}
\bibliography{di}
\end{document}
