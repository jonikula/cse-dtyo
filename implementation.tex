% Implementation Chapter

-- Implementation started with high-level requirements specification

-- Followed by some rapid prototyping as a PoC.

-- Desing refined and specified after first PoC.

-- Implementation.

-- Briefly on testing plan.


\section{Requirements specification}

\subsection{Product scope}
The software system will be a prototype of a model-based testing tool for the testers in ARM system testing team. It will be designed so that starting its use will be easy for the testers, requiring as little training as possible. It will include state-machine style modeling support, including relevant test selection criteria. 
The tool is designed so that it will be easily utilized as a part of the system testing teams existing tools. It will contain abstract interfaces for adapters used to generate the executable tests from abstract test cases generated by this tool. 

\subsection{Product perspective}
The tools developed for this thesis is a model-based testing tool designed to fit into the model-based testing process flow. It covers the first steps in the process: the model and the test case generator as well as an interface to enable use of test script generators which plug in to existing testing tools used by the organization.

\subsection{Product functionality, users and operating environment}
This section summarizes the major functions the software must perform or must let the the different actors perform through various use cases.

\subsubsection{Users and characteristics}
The normal user of the software is expected to be a test engineer in the ARM system testing team. They are expected to have basic skills in Python, experience in testing and formal modeling as well as a basic understanding of programming. 

The test engineer will be able to perform at least the following high-level use cases with the software. 

\begin{enumerate}
	\item Develop a model
	\item Maintain a model and related test suites
	\item Analyze requirements traceability matrix and coverage reports
	\item Add specific, custom test into the suite
\end{enumerate}  

-- Add brief descriptions of each use case

\subsubsection{Operating environment}
The main component of the tool will be a software application, which is usable in any modern operating system that has Python 2.7. The software is designed with a command line interface and so that it requires no external tools to be used. The tool will also need to be able to interact with current testing tools and adapters. 

\subsection{Design and implementation constraints}
There are very few constraints for this piece of software. Primarily it will need access to the file system to be able to generate test cases and read model files. The system also requires Python 2.7 or newer to run. The test case generation itself should not require access to internet or internal networks, but executing the test cases might require these. The tool should also be able to upload select parts of the files and data it generates into an online database. In addition to these minor constraints, the only major design constraint is the user interface. The command line interface needs to be taken into account in the design, as it places constraints on the usability of the system.

\subsection{User documentation}
All the major on minor functionalities will be documented in Markdown format in a GitHub repository, alogn with some easy-to-use tutorials and examples. UML documents describing the different parts of the system will be made available to the users as well as a part of this thesis.  


\subsection{Interface requirements}
-- short description

\subsubsection{User interfaces}

\subsubsection{Hardware interfaces}

\subsubsection{Software interfaces}

\subsubsection{Communications interfaces}

\subsection{Other non-functional requirements}

\subsubsection{Performance requirements}

\subsubsection{Safety and security requirements}

\subsubsection{Software quality attributes}




\section{Design}

\section{Implementation}

\section{Testing}