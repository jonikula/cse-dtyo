According to the Guide to the Software Engineering Body of Knowledge, the view of software testing has matured into a constructive one during the last few years. It is no longer seen as an activity that starts only after the coding phase is complete. Instead, it has become an activity that is pervasive throughout the development and maintenance cycle of the software \cite{swebok}. 

As the importance of testing grows, the methods and techniques used in testing need to evolve and grow as well. As the complexity and nature of the software under test grows, many previous methods of testing are becoming difficult to scale up at the same pace. Manual testing is too resource-intensive and different forms of capture and replay testing are too expensive to maintain with the modern, agile development processes. Many companies have transitioned into using automated testing with scripts executed periodically or on triggers by test harnesses. This is a logical step forward, but even this can sometimes be insufficient on its own, as the resource requirements for developing and maintaining these test scripts grows rapidly with the size of the software and the size of the test suite. Model-based testing pushes the level of automation further by automating the design, not just the execution, of the test cases \cite{tools}. 

Model-based testing is a test technique that has evolved from automated testing techniques via scripts and test harnesses. As the name suggests, model-based testing is built around the concept of a model, an abstract, formal representation of the SUT or of its software requirements. MBT is used to validate requirements, check the consistency of requirements and generate test cases focused on the behavioural aspects of the software. It is most often used in conjunction with test automation harnesses \cite{swebok}.

Some time ago a new architectural paradigm, Representational State Transfer (REST) was proposed and has since been widely adopted in the industry as an alternative or a replacement to the Service-Oriented Architecture (SOA). Systems that adopt this paradigm are called RESTful Web services \cite{richardson2008restful}. Although there are usually only four operations that RESTful web services can execute (PUT, POST, DELETE and GET) and only three of those operations alter the state of the system, the states and transitions of resources in a RESTful service can be complex. UML protocol state machines have been investigated for supporting the design and they can also be applied to testing with using model-based methods \cite{Pinheiro2013ModelBasedTO}.
